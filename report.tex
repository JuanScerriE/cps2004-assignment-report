\documentclass[12pt]{article}

% PACKAGES

\usepackage[
top=2.50cm,
bottom=2.50cm,
left=2cm,
right=2cm,
marginparsep=0pt,
marginparwidth=0pt]{geometry}
\usepackage{fancyhdr}
\usepackage{float}
\usepackage{dirtree}
\usepackage{cancel}
\usepackage{mathtools}
\usepackage{amsmath}
\usepackage{amsthm}
\usepackage{amssymb}
\usepackage{textcomp}
\usepackage{ulem}
\usepackage{verbatim}
\usepackage{contour}
\usepackage{graphicx}
\usepackage{svg}
\usepackage{xcolor}
\usepackage[T1]{fontenc}
\usepackage{inputenc}
\usepackage[utf8]{inputenx}
\usepackage[unicode]{hyperref}
\usepackage[shortlabels]{enumitem}
\usepackage{booktabs}
\usepackage{bookmark}
\usepackage{listings}
\usepackage{xcolor}
\usepackage{tocloft}
\usepackage{tikz}

% MACROS & DEFS

\newcommand{\floor}[1]{\left\lfloor #1 \right\rfloor}
\newcommand{\ceil}[1]{\left\lceil #1 \right\rceil}
\newcommand{\round}[1]{\left\lfloor #1 \right\rceil}
\newcommand{\abs}[1]{\left\lvert #1 \right\rvert}

\DeclareRobustCommand{\ul}[1]{%
	\uline{\phantom{#1}}%
	\llap{\contour{white}{#1}}%
}

\renewcommand{\ULdepth}{1.8pt}
\contourlength{0.8pt}

\setlength{\parindent}{0em}
\setlength{\parskip}{0.75em}

\definecolor{codegreen}{RGB}{0,135,0}
\definecolor{codegray}{RGB}{135,135,135}
\definecolor{codemagenta}{RGB}{215,0,135}
\definecolor{codepurple}{RGB}{135,0,175}
\definecolor{backcolour}{RGB}{238,238,238}

\def\cpp{{C\nolinebreak[4]\hspace{-.05em}\raisebox{.4ex}{\tiny\bf ++}}}

% PACKAGE CONFIG

% \graphicspath{ {./images/} }

\lstdefinestyle{code}{
	basicstyle=\ttfamily\small,
	commentstyle=\color{codegray}\itshape,
	keywordstyle=\color{codepurple},
	stringstyle=\color{codegreen},
	aboveskip=25pt,
    belowskip=10pt,
	breaklines=true,
	numbers=none,
	frame=tb,
	framesep=5pt,
	keepspaces=true,
	showspaces=false,
	showstringspaces=false,
	breakatwhitespace=false,
	tabsize=2,
	showtabs=false,
}

\lstset{style=code}

% Set dots for table of contents
\renewcommand{\cftdot}{.}
\renewcommand{\cftsecleader}{\cftdotfill{\cftdotsep}}

% Set theorem
\newtheorem*{definition}{Definition}

% HEADER & FOOTER

\setlength{\headheight}{15pt}
\pagestyle{fancy}
\renewcommand{\headrulewidth}{0pt}
\lhead{J. Scerri}
\chead{CPS2004 --- Assignment}
\rhead{\thepage}

% TITLE

\title{CPS2004 --- Object Oriented Programming\\
\vspace{1em}\textbf{Assignment}}

\date{\today}

\author {{\textbf{Juan Scerri}}\\
B.Sc. (Hons)(Melit.) Computing Science and Mathematics (Second Year)}

\begin{document}

%----------------------------------
%	TITLE PAGE
%----------------------------------

\maketitle % Print the title page

\thispagestyle{empty} % Suppress headers and footers on the title page

%----------------------------------

\tableofcontents

\clearpage

\listoffigures

\lstlistoflistings

\clearpage

\section{Plagiarism Declaration}

Plagiarism is defined as \textit{``the unacknowledged use, as
one's own, of work of another person, whether or not such work
has been published, and as may be further elaborated in Faculty
or University guidelines''} (\ul{University Assessment
Regulations}, 2009, Regulation 39 (b)(i), University of Malta).

I, the undersigned, declare that the report submitted is my
work, except where acknowledged and referenced. I understand
that the penalties for committing a breach of the regulations
include loss of marks; cancellation of examination results;
enforced suspension of studies; or expulsion from the degree
programme.

Work submitted without this signed declaration will not be
corrected, and will be given zero marks.

\vfill

\begin{minipage}[t]{0.3\textwidth}
\ul{Juan Scerri} \medskip

\textbf{Student's full name} \medskip
\end{minipage}
\hfill
\begin{minipage}[t]{0.3\textwidth}
\ul{CPS2004} \medskip

\textbf{Study-unit code} \medskip
\end{minipage}
\hfill
\begin{minipage}[t]{0.3\textwidth}
\ul{{\today}} \medskip

\textbf{Date of submission} \medskip
\end{minipage}

\vspace{2cm}

\textbf{Title of submitted work:} \ul{Object Oriented
Programming Assignment}

\vspace{2cm}

\textbf{Student's signature} \medskip

\underline{\includegraphics[height=2cm]{./images/sig.png}} \medskip

% Specifically:
% - explain language choice 
% - explain the user interface and how to interact with the game
% - mention the design (with uml) and design choices
% - important technical aspects which are crucial and
%   fundamental to the whole program
% - approach to testing and ensuring proper functionality of the
%   program
% - limitations and possible improvements

\section{Village War Game}

\subsection{Language Choice}
\subsection{User Guide}
\subsection{Design}
\subsection{Technical Aspects}
\subsection{Testing}
\subsection{Limitations \& Improvements}

\section{Minesweeper}

\subsection{Language Choice}

\cpp{} was chosen for Minesweeper because it has a fixed board
size of $16 \times 16$. This means that it is possible to stack
allocate every object removing the need for dynamic memory
allocation. This is facilitated by \texttt{std::array} from the
Standard Template Library (STL) which allows for the creation of
fixed size arrays on the stack.

\subsection{User Guide}

\subsubsection{Download, Compiling \& Running}

\begin{enumerate}
\item
    Clone the repository.
\begin{lstlisting}
$ git clone https://github.com/JuanScerriE/minesweeper
\end{lstlisting}

\item
    Compile the tests and the game.

    \textbf{Note:} Make sure that \texttt{gtest} and
    \texttt{ncurses} are installed for the tests and the game,
    respectively.
\begin{lstlisting}
$ cd minesweeper ; ./compile.sh
\end{lstlisting}

\item
    Run the tests.

    \textbf{Note:} Some tests might fail. This is because the
    sure that \texttt{gtest} and \texttt{ncurses} are installed
    for the tests and the game,
    respectively.
\begin{lstlisting}
minesweeper $ ./tests.sh
\end{lstlisting}

\item
    Run the game.
\begin{lstlisting}
minesweeper $ ./run.sh
\end{lstlisting}

\end{enumerate}

\subsubsection{Playing}

\subsection{Design}
\subsection{Technical Aspects}
\subsection{Testing}
\subsection{Limitations \& Improvements}

\end{document}
